

\section{SMPC Primitives}
\label{sec:smpc-prim}

\subsection{Oblivious Transfer}
\label{sec:ot-prim}

Oblivious transfer is a type of protocol in which a sender sends a potential subset of messages to the 
receiver but is oblivious as to whether which ones (if any) were received.

Michael Rabin~\cite{Rabin81:HtESwOT} introduced the first kind of oblivious 
transfer protocol, in which the sender sends a message with probability 
$\frac{1}{2}$, but is oblivious whether the receiver received it or not. A more
useful form of this protocol called the \emph{1-2 Oblivious Transfer} was
developed by Shimon Even, Oded Goldreich and Abraham Lempel~\cite{Even:1985:RPS:3812.3818}.
This protocol addresses the following problem: the sender has two messages 
$m_0$ and $m_1$, and the receiver wants one of the messages $m_b$, but the 
sender needs to remain oblivious about $b$, and the reciever needs to be 
oblivious about the value of $m_{\bar{b}}$.

\begin{algorithm}
\caption{On 1-2 oblivious transfer}
\label{algshare}
\begin{algorithmic}
\REQUIRE A has two messages, $m_0, m_1$, and wants to send exactly one of them to B, but does not want to know which B receives.
\STATE A generates a RSA key pair, comprising the modulus $N$, the public exponent $e$ and the private exponent $d$
\STATE A also generates two random values, $x_0, x_1$ and sends them to B 
along with the public modulus and exponent.
\STATE B picks $b$ to be either 0 or 1, and selects either the first or second $x_b$.
\STATE B generates a random value $k$ and blinds $x_b$ by computing $v = (x_b + k^e)\mod N$, which he sends to A.
\STATE A doesn't know which of $x_0$ and $x_1$ B chose, so she attempts to unblind with both of her random messages and comes up with two possible values for $k$: $k_0 = (v - x_0)^d\mod N$ and $k_1 = (v - x_1)^d\mod N$.  One of these will be equal to $k$ since it will correctly decrypt, while the other will produce another random value that does not reveal any information about $k$.
\STATE A blinds the two secret messages with each of the possible keys, $m'_0 = m_0 + k_0$ and $m'_1 = m_1 + k_1$, and sends them both to B.
\STATE B knows which of the two messages can be unblinded with $k$, so he is able to compute exactly one of the messages $m_b = m'_b - k$
\ENSURE each player $i$ has a share $v_{i}$ of the secret $v$
\end{algorithmic}
\end{algorithm}

\subsection{Shamir's Secret Sharing}
\label{sec:sss-prim}
This SMPC primitive addresses the following problem: suppose a group of
treasure hunters would like to lock a safe in such a way that it can't 
be opened unless there are atleast five (say) of them present at any given time. 
How many locks and keys would be required for this?

In~\cite{S79:HtSaS}, Shamir proposes a way of sharing a secret among $n$ 
players, such that any $k$ or more players can reconstruct the secret,
but no set of $k-1$ or less players can do so. This is called a $(k,n)$ 
{\it secret sharing scheme}, and is achieved by using $k-1$ degree polynomials
as described follows: 

\begin{algorithm}
\caption{On sharing a secret}
\label{algshare}
\begin{algorithmic}
\REQUIRE A player has a secret value $v$ which he has to share
\STATE select a random number $r$
\STATE $f(x) = v + r_{1}x + r_{2}x^{2} + \ldots + r_{k-1}x^{k-1}$
\FOR{all players $i$}
	\STATE send the value $v_{i}= f(i) = v + r_{1}i + r_{2}i^{2} + \ldots + r_{k-1}i^{k-1}$ to player $i$
\ENDFOR
\ENSURE each player $i$ has a share $v_{i}$ of the secret $v$
\end{algorithmic}
\end{algorithm}

The player who wishes to share a secret first chooses a $k-1$ degree
secret random polynomial (by choosing the $k-1$ coefficients $r_1$ to $r_k$),
say $f(x)$,
and sets the constant term to the value of the secret. He then calculates
the value of the ``share'' to be sent to each player $i$, as $f(i)$. With
this, it is ensured that each player has a ``share'' of the secret, which
he may reconstruct if and only if atleast $k-1$ other players are willing
to do so.

Notice, that
a $k-1$ degree polynomial's equation can be reconstructed with the knowledge
of any $k$ points on the curve (as in the case of any $k$ players colluding),
but any set of $k-1$ or less points will yield no information
about the equation of the curve (which means that any set of $k-1$ players
or less will not be able to reconstruct the secret!), and thus the objective
is achieved.

\subsubsection{Secret Addition}
\label{sec:add-prim}

\subsubsection{Secret Multiplication}
\label{sec:mult-prim}

\subsection{Privacy Preserving Union}
\label{sec:union-prim}
