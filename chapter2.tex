

\section{SMPC Primitives}
\label{sec:smpc-prim}

\subsection{Oblivious Transfer}
\label{sec:ot-prim}

\subsection{Shamir's Secret Sharing}
\label{sec:sss-prim}
In~\cite{S79:HtSaS}, Shamir proposes a way of sharing a secret among $n$ 
players, such that any $k$ or more players can reconstruct the secret,
but no set of $k-1$ or less players can do so. This is called a $(k,n)$ 
{\it secret sharing scheme}, and is achieved by using $k-1$ degree polynomials
as described follows: 

\begin{algorithm}
\caption{On sharing a secret}
\label{algshare}
\begin{algorithmic}
\REQUIRE A player has a secret value $v$ which he has to share
\STATE select a random number $r$
\STATE $f(x) = v + r_{1}x + r_{2}x^{2} + \ldots + r_{k-1}x^{k-1}$
\FOR{all players $i$}
	\STATE send the value $v_{i}= f(i) = v + r_{1}i + r_{2}i^{2} + \ldots + r_{k-1}i^{k-1}$ to player $i$
\ENDFOR
\ENSURE each player $i$ has a share $v_{i}$ of the secret $v$
\end{algorithmic}
\end{algorithm}

The player who wishes to share a secret first chooses a $k-1$ degree
secret random polynomial (by choosing the $k-1$ coefficients $r_1$ to $r_k$),
say $f(x)$,
and sets the constant term to the value of the secret. He then calculates
the value of the ``share'' to be sent to each player $i$, as $f(i)$. With
this, it is ensured that each player has a ``share'' of the secret, which
he may reconstruct if and only if atleast $k-1$ other players are willing
to do so.

Notice, that
a $k-1$ degree polynomial's equation can be reconstructed with the knowledge
of any $k$ points on the curve (as in the case of any $k$ players colluding),
but any set of $k-1$ or less points will yield no information
about the equation of the curve (which means that any set of $k-1$ players
or less will not be able to reconstruct the secret!), and thus the objective
is achieved.

\subsubsection{Secret Addition}
\label{sec:add-prim}

\subsubsection{Secret Multiplication}
\label{sec:mult-prim}

\subsection{Privacy Preserving Union}
\label{sec:union-prim}
