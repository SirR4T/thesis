

\section{SMPC Primitives}
\label{sec:smpc-prim}

\subsection{Oblivious Transfer}
\label{sec:ot-prim}

Oblivious transfer is a type of protocol in which a sender sends a potential subset of messages to the 
receiver but is oblivious as to whether which ones (if any) were received.

Michael Rabin~\cite{Rabin81:HtESwOT} introduced the first kind of oblivious 
transfer protocol, in which the sender sends a message with probability 
$\frac{1}{2}$, but is oblivious whether the receiver received it or not. A more
useful form of this protocol called the \emph{1-2 Oblivious Transfer} was
developed by Shimon Even, Oded Goldreich and Abraham Lempel~\cite{Even:1985:RPS:3812.3818}.
This protocol addresses the following problem: the sender has two messages 
$m_0$ and $m_1$, and the receiver wants one of the messages $m_b$, but the 
sender needs to remain oblivious about $b$, and the reciever needs to be 
oblivious about the value of $m_{\bar{b}}$.

\begin{algorithm}
\caption{On 1-2 oblivious transfer}
\label{alg12ot}
\begin{algorithmic}
\REQUIRE A has two messages, $m_0, m_1$, and wants to send exactly one of them to B, but does not want to know which B receives.
\STATE A generates a RSA key pair, comprising the modulus $N$, the public exponent $e$ and the private exponent $d$
\STATE A also generates two random values, $x_0, x_1$ and sends them to B 
along with the public modulus and exponent.
\STATE B picks $b$ to be either 0 or 1, and selects either the first or second $x_b$.
\STATE B generates a random value $k$ and blinds $x_b$ by computing $v = (x_b + k^e)\mod N$, which he sends to A.
\STATE A doesn't know which of $x_0$ and $x_1$ B chose, so she attempts to unblind with both of her random messages and comes up with two possible values for $k$: $k_0 = (v - x_0)^d\mod N$ and $k_1 = (v - x_1)^d\mod N$.  One of these will be equal to $k$ since it will correctly decrypt, while the other will produce another random value that does not reveal any information about $k$.
\STATE A blinds the two secret messages with each of the possible keys, $m'_0 = m_0 + k_0$ and $m'_1 = m_1 + k_1$, and sends them both to B.
\STATE B knows which of the two messages can be unblinded with $k$, so he is able to compute exactly one of the messages $m_b = m'_b - k$
\ENSURE each player $i$ has a share $v_{i}$ of the secret $v$
\end{algorithmic}
\end{algorithm}

In the algorithm~\ref{alg12ot}, we show one possible implementation of 
\emph{1-2 Oblivious Transfer} using the RSA cryptosystem~\cite{RSA78:AMfODSaPKC},
	but in general, the oblivious transfer protocol can be implemented using
	any ``trapdoor'' functions, such as one-way-functions and public key
	cryptosystems, or even Shamir's secret sharing~\cite{SSR08:APfGOT}.

\subsection{Shamir's Secret Sharing}
\label{sec:sss-prim}
This SMPC primitive addresses the following problem: suppose a group of
treasure hunters would like to lock a safe in such a way that it can't 
be opened unless there are atleast five (say) of them present at any given time. 
How many locks and keys would be required for this?

In~\cite{S79:HtSaS}, Shamir proposes a way of sharing a secret among $n$ 
players, such that any $k$ or more players can reconstruct the secret,
but no set of $k-1$ or less players can do so. This is called a $(k,n)$ 
{\it secret sharing scheme}, and is achieved by using $k-1$ degree polynomials
as described follows: 

\begin{algorithm}
\caption{On sharing a secret}
\label{algshare}
\begin{algorithmic}
\REQUIRE A player has a secret value $v$ which he has to share
\STATE select a random number $r$
\STATE $f(x) = v + r_{1}x + r_{2}x^{2} + \ldots + r_{k-1}x^{k-1}$
\FOR{all players $i$}
	\STATE send the value $v_{i}= f(i) = v + r_{1}i + r_{2}i^{2} + \ldots + r_{k-1}i^{k-1}$ to player $i$
\ENDFOR
\ENSURE each player $i$ has a share $v_{i}$ of the secret $v$
\end{algorithmic}
\end{algorithm}

The player who wishes to share a secret first chooses a $k-1$ degree
secret random polynomial (by choosing the $k-1$ coefficients $r_1$ to $r_k$),
say $f(x)$,
and sets the constant term to the value of the secret. He then calculates
the value of the ``share'' to be sent to each player $i$, as $f(i)$. With
this, it is ensured that each player has a ``share'' of the secret, which
he may reconstruct if and only if atleast $k-1$ other players are willing
to do so.


\begin{algorithm}
\caption{On reconstructing a secret}
\label{algrecon}
\begin{algorithmic}
\REQUIRE Each player has a share $v_{i}$ of the secret.
	$\mathcal{P}$ is the set of players who wish to reconstruct the secret.
\FOR{all players $i$}
	\STATE all the players send their shares to players in $\mathcal{P}$
\ENDFOR
\FOR{all players in $\mathcal{P}$}
	\STATE construct a polynomial $f(x) = a + bx$ which passes through all the points $(i,v_{i})$
	\STATE return the secret $v = a$
\ENDFOR
\ENSURE each player knows the secret $v$
\end{algorithmic}
\end{algorithm}

Notice, that
a $k-1$ degree polynomial's equation can be reconstructed with the knowledge
of any $k$ points on the curve (as in the case of any $k$ players colluding),
but any set of $k-1$ or less points will yield no information
about the equation of the curve (which means that any set of $k-1$ players
or less will not be able to reconstruct the secret!), and thus the objective
is achieved.

\subsubsection{Secret Addition}
\label{sec:add-prim}
Armed with the subroutines for secret sharing and reconstruction, we can
do additional things with it. Notice that the ``share''s are but 
numerical values, infact plot points on a polynomial curve. As with any 
polynomial, we can add, multiply, and do other arithmetic operations, but
what does adding polynomials mean, in this sense?

What we mean by doing arithmetic operations on the polynomial is the 
following:
Supposing two secrets $a$ and $b$ were shared by two different players
according to their choice of random private polynomials $f_{a}(x)$ and
$f_{b}(x)$. Let the shares of the $i$~th player be $a_i$ and $b_i$
respectively, of the two secrets. Now if the each of the individual players 
were to add their own shares, this would result in each of them creating
$c_i$, say, where $c_i = a_i + b_i$. But these individual shares 
would be plot points on the polynomial curve 
$f_{c}(x) = f_{a}(x) + f_{b}(x)$, which means that if the reconstruction
subroutine were to be run on these new shares, it would result in the 
reconstruction of the value $a+b$! And similar to the case of sharing 
a secret, even when the shares are added, the privacy of the secret still
remains intact, that is, all the players (except those two who shared $a$
and $b$) still have no information about the value of the individual secrets.

\begin{algorithm}
\caption{Computing Secure Addition}
\label{algadd}
\begin{algorithmic}
\REQUIRE $n$ is the number of players, each with private input $s_{i}$.
\STATE \COMMENT{Phase 1: Sharing}
\FOR{each player $i$}
	\STATE Share secret $s_{i}$
	\FOR{each player $j$}
		\STATE send $j$'s share $s_{ij}$ to player $j$
	\ENDFOR
\ENDFOR
\FOR{each player $j$}
	\STATE $sum_{j} = \sum_{i=1}^{n}s_{ij} $
\ENDFOR
\ENSURE each player is left with $sum_j$, the share of the sum of the private
inputs $s_{i}$.
\end{algorithmic}
\end{algorithm}

This process is illustrated in the algorithm~\ref{algadd}, 
where each of the
individual players share their private secrets, 
and add the individual shares.
This results in the addition of each of the secrets, 
so that at the end of this
routine, the group of individual players as a whole are 
each left with a share of
the total sum of the private inputs of all of the players.

\subsubsection{Secret Multiplication}
\label{sec:mult-prim}

For multiplication, we have a protocol given 
by~\cite{GRR98:SVFMCwAtTC}, which takes multiple rounds 
of communication, and 
is a little complex. Here, we present a simple and elegant protocol, 
with an initial preprocessing stage. The
preprocessing uses the protocol of~\cite{GRR98:SVFMCwAtTC}, but 
only once. All subsequent multiplications can be executed, using
the protocol~\ref{algmult}. The advantage with this protocol is 
that there is lesser communication complexity, 
and the computation can be done on the local shares.

We look at the simpler case of multiplying two secrets, 
$\alpha$ and $\beta$, 
which are assumed to be already shared among the $n$ players. 
In the initial preprocessing stage, 
we share two random numbers $x$ and $y$, and their product $xy$, 
while keeping the values themselves secret. 
The protocol given in ~\cite{GRR98:SVFMCwAtTC} is used to 
calculate $xy$. 

The numbers $\alpha$, $\beta$, $x$, $y$, $xy$ are now shared 
among the players. Now each player $i$ computes the values 
$\lambda_{i} =  \alpha_{i} - x_{i}$ and 
$\lambda_{i}' = \beta_{i} - y_{i}$, where $\eta_{i}$ stands
for the player $i$'s share of the value of $\eta$. Now the 
    values of $\lambda$ and $\lambda'$ are 
reconstructed, that is, made public among the players. This 
affords no information gain about the values
of $\alpha$ and $\beta$, as $x$ and $y$ are still private. 

Due to the linearity of the secret sharing protocol, we know 
that $\lambda = \alpha - x$ and $\lambda' = \beta - y$.
Also, 
\begin{eqnarray*}	
	\alpha\beta 	& = & (\alpha - x + x)(\beta - y + y) \\
			& = & (\alpha - x)(\beta - y) + y(\alpha - x) + x(\beta - y) + xy \\
			& = & \lambda\lambda' + y\lambda + x\lambda' + xy
\end{eqnarray*}

Since the respective shares of $x$, $y$ and $xy$ are already 
with the players, the computation of $\alpha\beta$
(rather the computation of the \emph{shares} of $\alpha\beta$) can 
be done locally. Since all the players have the shares of
the product, the required product can be reconstructed easily. 
This minimizes the cost of repeated multiplication,
which is very communication expensive. Also, the shares of $x$ 
and $y$ can be reused for all further multiplications,
without any loss of information.

\begin{algorithm}
\caption{Computing Secure Multiplication}
\label{algmult}
\begin{algorithmic}
\REQUIRE All the players have shares of random numbers $x$ and $y$ and their product $xy$.
They also have shares of $\alpha$ and $\beta$, of which they need to evaluate the product.
\STATE \COMMENT{Phase 1: SHARING}
\FOR{player $i$}
	\STATE  $\lambda_{i} = \alpha_{i} - x_{i}$
	\STATE $\lambda_{i}' = \beta_{i} - y_{i}$
	\STATE reconstruct the values $\lambda$ and $\lambda'$
	\STATE $\alpha\beta_{i} = \lambda\lambda' + y_{i}\lambda + x_{i}\lambda' + xy_{i}$
\ENDFOR
\end{algorithmic}
\end{algorithm}

\subsection{Privacy Preserving Union}
\label{sec:union-prim}

Owing to the protocols given above for secure addition and multiplication, we can rest assured that 
any algebraic function can be evaluated securely, since every function is but a series of repeated parametric
additions and multiplications, and their inverses (subtractions and divisions). Also, assuming that every player
can locally generate a random number, we can also safely say that every probabilistic polynomial time algorithm
can be executed securely by the players. Formal proofs of these claims exist in literature, but they are out of 
the scope of this paper. However, in addition to these protocols, we also present some specific purpose protocols
such as this one for Secure Union.

Let every player $i$ have a private set $S_{i}$, where $S_{i} \subset {S}$ is the universal set, of size $k$.
Without loss of generality, we assume that the set ${S}$ is ordered, and that ${s_{i}| 1 \leq i \leq k}$ are 
the elements. Each player creates a binary array, $A_{i} = [a_{ij}]$, where $a_{ij} = 1$ if $s_{j} \in S_{i}$ and $0$ otherwise.
Now it remains to logically OR the arrays $A_{i}$ bitwise, and create the union set $U$. Note that, in binary operations, 
the AND gate corresponds to multiplication, and the XOR gate corresponds to addition. We have already described protocols
for multiplication and addition. The only difference here will be that the coefficients, operands and operations are in 
binary logic. Also, the OR gate can be decomposed to the AND and XOR gate. And since we know the protocols for these
gates, the OR gate can also be computed. The algorithm for the construction of secure union set is given below.

\begin{algorithm}
\caption{Constructing Secure Union}
\label{algonion}
\begin{algorithmic}
\REQUIRE player $i$ has the array $A_{i} = [a_{ij}]$ which represents the set $S_{i}$
\FOR{bit $j$ from $j = 1$ to $k$}
	\FOR {all players}
		\STATE share bit $a_{ij}$ in binary \COMMENT{Use the protocol~\ref{algadd}}
		\STATE construct bit $XORa_{j} = {XOR}_{t=1}^{n}a_{ij}$ using protocol~\ref{algadd}
		\STATE construct bit $ANDa_{j} = {AND}_{t=1}^{n}a_{ij}$ using protocol~\ref{algmult}
		\STATE construct bit $ORa_{j} = XORa_{j} XOR ANDa_{j}$ using protocol~\ref{algadd}
		\STATE element $u_{j} = ORa_{j}$
	\ENDFOR
\ENDFOR
\STATE return set $U = \{u_{j}|j = 1 $ to $ k\}$
\end{algorithmic}

\end{algorithm}

The protocol for secure intersection is similar to this, where instead of constructing the XOR of the 
bits, just an AND of the bits is enough. Which is to say that a modified multiplication protocol will 
be the same as a secure intersection protocol. 


\subsection{Secure Maximisation}
\label{sec:max}

We suppose that the $n$ players each have a private input value, a probability, in the range $[0,1]$, and that
they wish to find out the maximum value amongst them, and the player who holds it. The protocol mimics the binary 
search over a given range. First, the interval is divided into two halves, $[0,0.5]$ and $[0.5,1]$, and the number of
players in the greater interval are counted. Then, the interval with the largest value is subdivided into two halves,
and the same process is repeated recursively, until only one player remains in the greater interval. This 
value is then shared.

\begin{algorithm}
\caption{Maximisation of probabilities}
\label{algmax}
\begin{algorithmic}
\REQUIRE each player has a private value $p_{i}$ within the range $[0,1]$
\STATE $A = 0, B = 1$
\LOOP
\STATE $C = (A + B) / 2$
\FOR{each player $i$}
	\STATE $a_{i} = 0 $; $b_{i} = 0$
	\IF{value $p_{i} < C$ }
		\STATE $a_{i} = 1 $
	\ELSE
		\STATE $ b_{i} = 1$
	\ENDIF
	\STATE \COMMENT{Use the Secure Addition protocol~\ref{algadd}}
	\STATE add all players' private $a_{i}$'s and $b_{i}$'s, call them $a$ and $b$
	\STATE reconstruct $a$ and $b$
	\IF{$b \geq 1$}
		\STATE $A = C$ and continue loop
	\ELSIF{$b = 1$}
		\STATE exit loop \COMMENT{Player with max value found}
	\ELSIF{$a \geq 1$}
		\STATE $B = C$ and continue loop
	\ELSE
		\STATE exit loop \COMMENT{Player with max value found}
	\ENDIF
\ENDFOR
\ENDLOOP
\STATE \COMMENT{All players know for themselves whether they are or not the one with the max value}
\FOR{all players $i$}
	\IF{$a_{i} = 1$ or $b_{i} = 1$}
		\STATE share value $p_{i}$
	\ELSE
		\STATE share $0$
	\ENDIF
\ENDFOR
\end{algorithmic}
\end{algorithm}



All the above mentioned algorithms and protocols end with the actual answer being shared. 
For the answer to be revealed, one can always run the reconstruction algorithm (algorithm~\ref{algrecon}).


